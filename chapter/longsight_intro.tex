\section{LongSight Introduction}


LongSight is project developed by Wenyin and Tingzhi who devoted to develop a system which is robust to handle different real scenes and reconstruct corresponding electronic point cloud.


\begin{enumerate}
    \item \textbf{Adopted Scheme}\\
    现行方案是采用分布式系统,一个单片机对一个摄像头(未来可能一个 FPGA 对多个摄像头),对体模重建任务,六摄像头应该是个合理的估计。

    分布式系统采用 nsq 作为消息传递机制。
    \begin{itemize}
        \item Github: https://github.com/nsqio/nsq
        \item Official Website: https://nsq.io/
    \end{itemize}



    \item \textbf{Cases <= 4 Cameras}\\
    4 个摄像头以下,推荐使用 PCI-E 转 USB 扩展卡接电脑台式机主机 PCI 插槽。
    优点:
    \begin{enumerate}
        \item  支持 1600 万像素的摄像头,带宽比视频录像机单路高不少。
        \item  成本低,不需要硬盘录像机,直接连 PC 上,程序可以同步处理。
    \end{enumerate}

    缺点:
    \begin{enumerate}
        \item 需要主机,移动不方便。
        \item 摄像头数量达到 5 路以上不太建议使用这种方式,对动态场景 4 路摄像头大概率不够。
    \end{enumerate}

    成本:(PCI-e 转接头 100¥ + 1600 万像素摄像头 850 * 4) =3500 ¥,未计入主机。

    \item \textbf{Cases > 4 Cameras}\\
    5 个摄像头以上,基本排除其他可能,只能用录像机作为记录视频主要工具。这是我们第一步要考虑的情况。
    优点:
    \begin{enumerate}
        \item 1. 记录视频的专业设备,单路带宽虽然有限 (800 万像素) 但很稳定,不容易掉帧。
        \item 2. 不管多少路都可以支持。
    \end{enumerate}

    
    缺点:
    \begin{enumerate}
        \item 1. 需要新购录像机 2000¥。
        \item 2. 不同厂商的硬盘录像机系统不大一样,用网线进行和 PC 的连接之后可以有线下载视频文件到本地进行处理。
        \href{https://iask.sina.com.cn/b/6gqyGEWCCKd.html}{硬盘录像机设置怎么才能跟电脑连接?}
    \end{enumerate}

    成本:八类网线 40¥ + Type-C 转万兆网卡 129¥ + 录像机+八路800万像素摄像头约计 6000¥,总成本估计 6500¥ 以内

    \item \textbf{Official Solution}\\   
    海康的机器视觉团队也有自己做的多路方案,产品展示有一张图片出现过,需要用到二次开发,我接下来会进一步调研。
    \href{https://www.hikvision.com/cn/download_61.html}{海康视觉的 SDK 开发套件}
    \href{https://www.hikrobotics.com/vision/visionlist.htm}{海康机器视觉官网}
\end{enumerate}

